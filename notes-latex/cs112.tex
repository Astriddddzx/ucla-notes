\documentclass{scrartcl}

\KOMAoptions{
  fontsize=8pt,
}

\addtokomafont{disposition}{\rmfamily}

\usepackage{bm}
\usepackage{graphicx}
\usepackage{mathtools}
\usepackage{amsmath}
\usepackage{float}
\usepackage{ragged2e}
\usepackage[font=footnotesize,labelfont=bf]{caption}
\usepackage{geometry}
\geometry{letterpaper, left=.1in, right=.1in, top=.05in, bottom=.05in}
\usepackage{multicol}

\DeclareOldFontCommand{\bf}{\normalfont\bfseries}{\mathbf}

\pagenumbering{gobble}
\setlength{\parskip}{0.2em}
\setlength\parindent{0pt}

\begin{document}
\raggedright
\begin{multicols*}{3}



{\bf Taylor Series}

$\frac{1}{x} = \sum_{n=0}^{\infty}x^n$

$e^x = \sum_{n=0}^{\infty}\frac{x^n}{n!}$

$\cos x = \sum_{n=0}^{\infty}(-1)^n\frac{x^{2n}}{(2n)!}$

$\sin x = \sum_{n=0}^{\infty}(-1)^n\frac{x^{2n+1}}{(2n+1)!}$

$\ln(1+x) = \sum_{n=1}^{\infty}(-1)^{n+1}\frac{x^n}{n}$



{\bf Permutations and Combinations}

$P(n,k)=\frac{n!}{(n-k)!}$

$C(n,k) = {n\choose k} = \frac{n!}{(n-k)!k!}$



{\bf Laplace Transforms}

$F^*(s)=\int_0^\infty f(t)e^{-st}dt$

$f(t) = \int_0^\infty F^*(s)e^{st}ds$



{\bf Convolution Property}

$f(t)*g(t)=\int_0^t f(t-x)g(x)dx \leftrightarrow F^*(s)G^*(s)$



{\bf Z-Transform}

Mapping of discrete function $f_n$ into complex fuction with variable $z$.

$F(z)=\sum_{n=0}^{\infty}f_nz^n$



{\bf Probability and Conditional}

$P(A\cup B) = P(A) + P(B) - P(A\cap B)$

$P(A|B) = \frac{P(A\cap B)}{P(B)}$

$A$, $B$ are independent if $P(A,B)=P(A)P(B)$



{\bf Total Probability}

$P(B) = \sum_{i} P(A_i)P(B|A_i)$



{\bf Bayes' Rule}

$P(A_i|B) = \frac{P(A_i\cap B)}{P(B)} = \frac{P(A_i)P(B|A_i)}{P(B)} = \frac{P(A_i)P(B|A_i)}{\sum_jP(A_j)P(B|A_j)}$



{\bf PMF (Probability Mass Function)}

$p_X(x)=p(\{s\in \Omega\text{ s.t. }X(s)=x\})$

$\sum_xp_X(x)=1$



{\bf Bernoulli Random Variable}

$X=1$ on success, $X=0$ on failure.

$p(X=x) = p$, if $x=1$

$p(X=x) = 1-p$, if $x=0$



{\bf Geometric Random Variable}

Counts \#trials until first success.

$p_X(x)=(1-p)^{x-1}p$, $x=1,2,\cdots$

$p(X\geq s+1|X\geq t)=p(X\geq s)$



{\bf Binomial Random Variable}

Counts \#success in $n$ identical independent experiments.

$p_X(x)={n\choose x}p^x(1-p)^{n-x}$, when $0\leq x\leq n$

$p_X(x)=0$, otherwise



{\bf Poisson Random Variable}

Model occurrence of event over time interval assuming event happens at rate $\lambda$

$p_X(x)=e^{-\lambda}\frac{\lambda^x}{x!}$, when $x=0,1,\cdots$



{\bf PDF (Probability Density Function)}

$\int_{-\infty}^\infty f_X(x)dx=1$



{\bf CDF (Cumulative Distribution Function)}

$F_X(x)=P(X\leq x)$

$\lim_{x\to -\infty}F_X(x)=0$

$\lim_{x\to\infty}F_X(x)=1$

$P(a<X\leq b)=F_X(b)-F_X(a)$



{\bf Uniform Distribution}

$f_X(x)=\frac{1}{b-a}$, when $a\leq x\leq b$

$f_X(x)=0$, otherwise



{\bf Exponential Distribution}

Memoryless continuous distribution.

$f_X(x)=\lambda e^{-\lambda x}$, when $x\geq 0$

$F_X(x)= 1-e^{-\lambda x}$, when $x\geq 0$

$F_X(x)=0$, otherwise

$P(X>x)=e^{-\lambda x}$



{\bf Expectation}

$E[X] = \sum_xxp(x)$

$E[X] = \int_{-\infty}^{\infty}xf_X(x)dx$

If $Y=g(X)$, $E[Y] = \sum_xg(x)p(x)$, $E[Y] = \int_{-\infty}^{\infty}g(x)f_X(x)dx$

$E[X+Y] = E[X] + E[Y]$

$E[aX] = aE[X]$

$E[XY] = E[X]E[Y]$, if $X$,$Y$ are independent.

For $X$, $Y$ with joint PMF $p(x,y)$ or PDF $f_{X,Y}(x,y)$, $E[XY] = \sum_{(x,y)}xyp(x,y)$, $E[XY] = \int_{-\infty}^{\infty}\int_{-\infty}^{\infty}xyf_{X,Y}(x,y)dxdy$



{\bf Conditional Expectation}

$X$,$Y$ are random variables, $E[Y|X] = \sum_yyP(Y=y|X=x)=\sum_yyp_{Y|X}(y|x)$, $E[Y|X] = \int_{-\infty}^{\infty}yf_{Y|X}(y|x)dy$



{\bf Unconditional Expectation}

$E[Y] = \sum_xE[Y|X]p_X(x)$

$E[Y] = \int_{\infty}^{\infty}E[Y|X]f_X(x)dx$



{\bf Variance}

$Var[X] = E[(X-E[X])^2] = \sum_x(x-E[X])^2p(x) = \int_{\infty}^{\infty}(x-E[X])^2f_X(x)dx$

$Var[X]=E[X^2]-E[X]^2$

$Var[X+Y]=Var[X]+Var[Y]$, if $X$,$Y$ are independent.



{\bf Expectations and Variances}

Binomial: $np$, $np(1-p)$

Geometric: $\frac{1}{p}$, $\frac{1-p}{p^2}$

Uniform: $\frac{a+b}{2}$, $\frac{(b-a)^2}{12}$

Exponential: $\frac{1}{\lambda}$, $\frac{1}{\lambda^2}$

Poisson: $\lambda$, $\lambda$



{\bf Covariance}: measure of joint probability

$Cov(X,Y) = E[(X-E[X])(Y-E[Y])]$

$Cov(X,Y) = E[XY] - E[X]E[Y]$

If $X$,$Y$ are independent, $Cov(X,Y)=0$



{\bf Correlation}: scaled version of covariance

$\rho(X,Y)=\frac{Cov(X,Y)}{\sqrt{Var(X)Var(Y)}}$, range $[-1,1]$



{\bf CTMC}: State transitions are permitted at arbitrary time instances. The amount of time spent in a state is exponentially distributed.

{\bf State Transition Probability}: $p_{ij}(t) = p(X(\tau+t)=j|X(\tau)=i)$

{\bf Chapman-Kolmogorov Equation}: $p_{ij}(s+t) = \sum_kp_{ik}(s)p_{kj}(t)$

{\bf Transition Probability $H(t)$}:
$H(t) = \{p_{ij}(t)\}$ \\
$H(s+t) = H(s)H(t)$ \\
$H(t+\Delta t) = H(t)H(\Delta t)$ \\
$H(t+\Delta t) = H(t)[H(\Delta t)-I]$ \\
$\frac{dH(t)}{dt} = H(t)\lim_{\Delta t\to 0}[\frac{H(\Delta t) - I}{\Delta t}]$ \\
$Q = \lim_{\Delta t\to 0}[\frac{H(\Delta t)-I}{\Delta t}]$ \\
$\frac{dH(t)}{dt} = H(t)Q$

{\bf Transition Rate Matrix}: $Q$, infinitesimal generator of $H(t)$. $H(t)=e^{Qt}$.

Diagonal elements $\leq 0$. $q_{ii}=\lim_{\Delta t\to 0}[\frac{p_{ii}(\Delta t)-1}{\Delta t}]$.

Off-diagonal elements $\geq 0$. $q_{ij}=\lim_{\Delta t\to 0}[\frac{p_{ij}(\Delta t)-0}{\Delta t}]$, for $i\neq j$.

In each row, sum of off-diagonal = magnitude of diagonal: $q_{ii}=-\sum_{i\neq j}q_{ij}$

{\bf State Probabilities}:\\ $\pi(t)=\pi(0)H(t)$, $\;\pi(t)=\pi(0)e^{Qt}$

{\bf Stationary Distribution}: $\pi Q=0$, $\;\sum_j\pi_j=1$



{\bf ==========================}

% QUESTIONS


{\bf Q1}: A given program has an execution time that is uniformly distributed between 10 and 20 seconds. The number of interrupts that occur during execution is a Poisson random variable with parameter λt where t is the program execution time. The probability distribution of the number of interrupts is therefore $P(N = k) = (\lambda t)ke^{−\lambda t}$.

(a) What is $E[N|T = t]$, where $N$ is the number of interrupts the program experiences, and $T$ is the running time of the program.

$E[N|T = t] = \lambda t$, since for fixed running time, the number of interrupts is a Poisson random variable with mean $\lambda t$.

(b) Find the expected number of interrupts the program experiences during a randomly
selected execution.

$E[N] = \int_{10}^{20} E[N|T=t]f_T(t)dt = \int_{10}^{20}\frac{\lambda t}{10}dt = 15\lambda$



{\bf Q2}: Suppose that you made a webpage and you are collecting the statistics from the visitors. There are $m$ types of visitors. Each visit is equally likely to be any of the $m$ types. Find the expected number of visitors needed in order to have at least one of each type. Hint: Let $X$ denote the number of visitors needed. It is useful to represent $X$ by $X=\sum_{i=1}^mX_i$ where each $X_i$ is a geometric random variable.

Suppose the current visitor pool contains $i$ different types. Let $X_i$ denote the number of additional visitors needed until it contains $i + 1$ types. The $X_i$ is are independent geometric random variables with parameter $(m - i)/m$, $i = 0, 1, \cdots, m-1$. $E[X]=E[\sum_{i=1}^mX_i]=\sum_{i=1}^mE[X_i]=\sum_{i=1}^m\frac{m}{m-i}$



{\bf Q3}: A Markov chain $\{X_n,n \geq 0\}$ with states 0, 1, 2, has the transition probability matrix $\begin{bmatrix}\frac{1}{2} & \frac{1}{3} & \frac{1}{6} \\ 0 & \frac{1}{3} & \frac{2}{3} \\ \frac{1}{2} & 0 & \frac{1}{2}\end{bmatrix}$ If $P[X_0 = 0] = P[X_0 = 1] = 1$, find the state probability vector $P[X_3 = 2]$.

Cubing the transition probability matrix, we obtain $P^3 = \begin{bmatrix}\frac{13}{36} & \frac{11}{54} & \frac{47}{108} \\ \frac{4}{9} & \frac{4}{27} & \frac{11}{27} \\ \frac{5}{12} & \frac{2}{9} & \frac{13}{36}\end{bmatrix}$
$P[X_3=2]=\frac{1}{4}\cdot\frac{47}{108}+\frac{1}{4}\cdot\frac{11}{27}+\frac{1}{2}\cdot\frac{13}{36}$



{\bf Q4}: A workstation tries to transmit frames through Ethernet. Suppose that whether or not collision occurs in the current transmission depends on the result of the last two trans- missions the workstation had. That is, suppose that if collisions have occurred in both of the past two transmissions, then with probability 0.7 a collision will occur in the current transmission; if a collision occurs in last transmission but not the transmission before the last one, then a collision will occur in the current transmission with probability 0.5; if a collision occurred in the transmission before the last one but not the last one, then one will occur in the current transmission with probability 0.4; if there have been no collision in the past two transmissions, then a collision will occur in the current transmission with probability 0.2. (Hint: Note that the state description needs to include status of last two transmissions).

(b) Find the transition probability matrix.

$P=\begin{bmatrix}0.7 & 0 & 0.3 & 0 \\ 0.5 & 0 & 0.5 & 0 \\ 0 & 0.4 & 0 & 0.6 \\ 0 & 0.2 & 0 & 0.8\end{bmatrix}$

(c) What fraction of frames suffer a collision?

Solve $\pi = \pi P$ to obtain the stationary state probabilities. Then the fraction of frames suffering a collision is $\pi_0 + \pi_2$ (or $\pi_0 + \pi_1$).

\end{multicols*}
\end{document}
